%
% File acl2015.tex
%
% Contact: car@ir.hit.edu.cn, gdzhou@suda.edu.cn
%%
%% Based on the style files for ACL-2014, which were, in turn,
%% Based on the style files for ACL-2013, which were, in turn,
%% Based on the style files for ACL-2012, which were, in turn,
%% based on the style files for ACL-2011, which were, in turn, 
%% based on the style files for ACL-2010, which were, in turn, 
%% based on the style files for ACL-IJCNLP-2009, which were, in turn,
%% based on the style files for EACL-2009 and IJCNLP-2008...

%% Based on the style files for EACL 2006 by 
%%e.agirre@ehu.es or Sergi.Balari@uab.es
%% and that of ACL 08 by Joakim Nivre and Noah Smith

\documentclass[11pt]{article}
\usepackage{acl2015}
\usepackage{times}
\usepackage{url}
\usepackage{latexsym}
\usepackage{hyperref}
\usepackage{tikz}
\usepackage{amsmath}

\usepackage[labelsep=quad,indention=10pt]{subfig}
\captionsetup*[subfigure]{position=bottom}

\newcommand{\specialcell}[2][c]{%
  \begin{tabular}[#1]{@{}c@{}}#2\end{tabular}}

\usepackage{graphicx}
\graphicspath{{figures/}}
\DeclareGraphicsExtensions{.eps,.pdf,.jpg,.png}

\DeclareMathOperator{\wsim}{sim}

%\setlength\titlebox{5cm}

% You can expand the titlebox if you need extra space
% to show all the authors. Please do not make the titlebox
% smaller than 5cm (the original size); we will check this
% in the camera-ready version and ask you to change it back.

\title{Title}

\author{J.T. Cho \\
	{\tt joncho@} \\
	{\tt seas.upenn.edu} \\\And
	Karinna Loo \\
	{\tt kloo@} \\
	{\tt seas.upenn.edu} \\\And
  	Veronica Wharton \\
  	{\tt whartonv@} \\
	{\tt seas.upenn.edu} }
\date{}

\begin{document}
\maketitle

\begin{abstract}

\noindent TODO: Abstract 

\end{abstract}

\section{Introduction}

For our CIS 625 final project, our team --- JT Cho, Karinna Loo, and Veronica Wharton --- took a closer look at the topic of fairness in machine learning. The paper that piqued our interest was \textit{Rawlsian Fairness for Machine learning} \cite{DBLP:journals/corr/JosephKMNR16}, which discusses the procedures of studying the quantifying the discriminatory behaviors in automated decision-making procedures and proposing algorithms that both learn at a rate comparable to (but necessarily worse than) the best algorithms absent of a fairness constraint and also satisfy a specified fairness constraint. Specifically, our team was interested in exploring the paper's results via implementation, as well as exploring what further applications for fairness analysis in real-world data might be possible. 

\noindent TODO: Problem overview

\noindent TODO: Potential applications

\noindent TODO: Literature review, including overview of \newcite{DBLP:journals/corr/JosephKMNR16}

\section{Project overview}

Our project consisted of the following steps:

\begin{enumerate}
	\item We read the paper \textit{Rawlsian Fairness for Machine Learning}  \cite{DBLP:journals/corr/JosephKMNR16}.
	\item We implemented the TopInterval, IntervalChaining, and RidgeFair algorithms from the paper in Python. 
	\item We ran our implementations on a Yahoo! dataset containing a fraction of the user click log for news articles displayed in the Featured Tab of the Today Module on the Yahoo! Front Page during the first ten days in May 2009, to see how well they performed on real data.
	\item To empirically evaluate our implementations, we ran experiments similar to those in \cite{DBLP:journals/corr/JosephKMNR16} with randomly-drawn contexts. 
	\item We compiled our findings into a written report.
\end{enumerate}

\section{Implementation: IntervalChaining}

\section{Implementation: RidgeFair}

\section{Experimental results: generated data}

\noindent TODO: Pretty figures

\section{Experimental results: real data}

\noindent TODO: Pretty figures

\section{Conclusion}

\bibliography{paper}
\bibliographystyle{acl}

\end{document}